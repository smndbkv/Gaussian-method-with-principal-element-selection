\documentclass[a4paper,12pt]{article}
\usepackage[utf8]{inputenc}
\usepackage[english,russian]{babel}
\usepackage[T2A]{fontenc}

\usepackage[
  a4paper, mag=1000, includefoot,
  left=1.1cm, right=1.1cm, top=1.2cm, bottom=1.2cm, headsep=0.8cm, footskip=0.8cm
]{geometry}

\usepackage{amsmath}
\usepackage{amssymb}
\usepackage{times}
\usepackage{mathptmx}
\usepackage{algorithm}
\usepackage{algpseudocode}
\usepackage{xcolor}
\usepackage{geometry}

\IfFileExists{pscyr.sty}
{
  \usepackage{pscyr}
  \def\rmdefault{ftm}
  \def\sfdefault{ftx}
  \def\ttdefault{fer}
  \DeclareMathAlphabet{\mathbf}{OT1}{ftm}{bx}{it} % bx/it or bx/m
}

\mathsurround=0.1em
\clubpenalty=1000%
\widowpenalty=1000%
\brokenpenalty=2000%
\frenchspacing%
\tolerance=2500%
\hbadness=1500%
\vbadness=1500%
\doublehyphendemerits=50000%
\finalhyphendemerits=25000%
\adjdemerits=50000%

\begin{document}

\author{Дубков Семён}
\title{Метод Гаусса решения линейной системы с выбором главного элемента по всей матрице}
\date{\today}
\maketitle

\section{Постановка задачи}

Рассмотрим систему $AX=B$ с матрицей
$$A=
   \begin{pmatrix}
     a_{11}& a_{12} &\ldots & a_{1n}\\
     a_{21}& a_{22} &\ldots & a_{2n}\\
     \vdots& \vdots &\ddots & \vdots\\
     a_{n1}& a_{n2} &\ldots & a_{nn}
    \end{pmatrix}
$$
и правой частью
$$B=
   \begin{pmatrix}
    b_{1} \\
     \vdots \\
     b_{n}
    \end{pmatrix}.
$$

Представим матрицу в виде:
$$A=
  \begin{pmatrix} 
    A_{11}^{m \times m} & A_{12}^{m \times m} & \cdots & A_{1,k}^{m \times m} & A_{1,k+1}^{m \times l} \\
    A_{21}^{m \times m} & A_{22}^{m \times m} & \cdots & A_{2,k}^{m \times m} & A_{2,k+1}^{m \times l} \\ 
    \vdots & \vdots & \ddots & \vdots & \vdots \\ 
    A_{k,1}^{m \times m} & A_{k,2}^{m \times m} & \cdots & A_{k,k}^{m \times m} & A_{k,k+1}^{m \times l} \\
    A_{k+1,1}^{l \times m} & A_{k+1,2}^{l \times m} & \cdots & A_{k+1,k}^{l \times m} & A_{k+1,k+1}^{l \times l} 
  \end{pmatrix}.
$$
Правую часть представим в виде:
$$B=
   \begin{pmatrix}
    B_{1}^{m\times 1} \\
     \vdots \\
     B_{k}^{m\times 1}\\
     B_{k+1}^{l\times 1}
    \end{pmatrix}.
$$
Исходную матрицу будем хранить в массиве стандартным образом:
$$
a[(i-1) \cdot n + (j-1)] = a_{ij},\quad b[(i-1)]=b_i, \quad i=1,\dots,n,\quad j=1,\dots,n.
$$

\newpage
\section{Функции get\_block(), set\_block()}

\begin{algorithm}
\caption{Функция \texttt{get\_block(n, m, i, j, a, c, v, h)}}
\begin{algorithmic}[1]
\State // Возвращает блок $A_{ij}$ в матрицу $C$
\State // $v$, $h$ — размеры матрицы $C$
\State $k = n / m$
\State $l = n \% m$
\State $v = (i < k) ? m : l$
\State $h = (j < k) ? m : l$
\For{$r = 0$ to $v-1$}
    \For{$s = 0$ to $h-1$}
        \State $c[r \cdot h + s] = a[(i \cdot m + r) \cdot n + (j \cdot m + s)]$
    \EndFor
\EndFor
\end{algorithmic}
\end{algorithm}

\begin{algorithm}
\caption{Функция \texttt{set\_block(n, m, i, j, a, c, v, h)}}
\begin{algorithmic}[1]
\State // Устанавливает значения блока $A_{ij}$ из матрицы $C$
\State // $v$, $h$ — размеры матрицы $C$
\State $k = n / m$
\State $l = n \% m$
\State $v = (i < k) ? m : l$
\State $h = (j < k) ? m : l$
\For{$r = 0$ to $v-1$}
    \For{$s = 0$ to $h-1$}
        \State $a[(i \cdot m + r) \cdot n + (j \cdot m + s)] = c[r \cdot h + s]$
    \EndFor
\EndFor
\end{algorithmic}
\end{algorithm}

\newpage
\section{Описание алгоритма}

На $s$-ом шаге алгоритма ищем блок с минимальной нормой обратной матрицы $B_{ij} = A_{ij}^{-1}$ (главный элемент):
$$
\|B_{ij}\| = \max_{0 \leq r < m} \sum_{t=0}^{m-1} |b_{(i \cdot m + r) \cdot n + j \cdot m + t}|,\ i = s,\dots k, j =s,\dots k.
$$
Пусть $A_{i_0j_0}$ — главный элемент. Переставим его на позицию $A_{ss}$, переставляя строки и столбцы в матрице и строки в правой части. Запоминаем в массиве перестановку столбцов $p[s] = j_0$. На $s$-ом шаге получаем матрицу, где $T$ — верхняя треугольная блочная матрица:
$$A=
  \begin{pmatrix} 
    \ T & \cdots & \cdots & \cdots & \cdots \\
    0  & A_{ss}^{m \times m} & \cdots & A_{s,k}^{m \times m} & A_{s,k+1}^{m \times l} \\ 
    \vdots & \vdots & \ddots & \vdots & \vdots \\ 
    \vdots& A_{k,2}^{m \times m} & \cdots & A_{k,k}^{m \times m} & A_{k,k+1}^{m \times l} \\
    0 & A_{k+1,2}^{l \times m} & \cdots & A_{k+1,k}^{l \times m} & A_{k+1,k+1}^{l \times l} 
  \end{pmatrix},\ 
  B=
  \begin{pmatrix}
  \vdots \\
  B_{s}^{m\times 1}\\
  \vdots\\
  B_{k+1}^{l\times 1}
  \end{pmatrix}.
$$
Находим обратную матрицу $C^{m\times m} = A^{-1}_{ss}$ и домножаем строку и правую часть на неё слева:
$$A=
  \begin{pmatrix} 
    \ T & \cdots & \cdots & \cdots & \cdots \\
    0  & E & \cdots & C^{m\times m}A_{s,k}^{m \times m} & C^{m\times m}A_{s,k+1}^{m \times l} \\ 
    \vdots & \vdots & \ddots & \vdots & \vdots \\ 
    \vdots& A_{k,2}^{m \times m} & \cdots & A_{k,k}^{m \times m} & A_{k,k+1}^{m \times l} \\
    0 & A_{k+1,2}^{l \times m} & \cdots & A_{k+1,k}^{l \times m} & A_{k+1,k+1}^{l \times l} 
  \end{pmatrix},\ 
  B=
  \begin{pmatrix}
  \vdots \\
  C^{m\times m}B_{s}^{m\times 1}\\
  \vdots\\
  B_{k+1}^{l\times 1}
  \end{pmatrix}.
$$
Затем обнуляем столбец, используя элементарные преобразования строк:
$$
\widetilde{A}_{ij} = A_{ij} - A_{is}(CA_{sj}),\quad \widetilde{B}_i=B_i - A_{is}(CB_s),\quad i =s+1,\dots k+1,\ j=s+1,\dots k+1.
$$
Получаем матрицу:
$$A=
  \begin{pmatrix} 
    \ \widetilde{T} & \cdots & \cdots & \cdots & \cdots \\
    0  & \widetilde{A}_{s+1,s+1}^{m\times m} & \cdots &\widetilde{A}_{s+1,k}^{m \times m} & \widetilde{A}_{s+1,k+1}^{m \times l} \\ 
    \vdots & \vdots & \ddots & \vdots & \vdots \\ 
    \vdots& \widetilde{A}_{k,2}^{m \times m} & \cdots & \widetilde{A}_{k,k}^{m \times m} & \widetilde{A}_{k,k+1}^{m \times l} \\
    0 & \widetilde{A}_{k+1,2}^{l \times m} & \cdots & \widetilde{A}_{k+1,k}^{l \times m} & \widetilde{A}_{k+1,k+1}^{l \times l} 
  \end{pmatrix},\ 
  B=
  \begin{pmatrix}
  \vdots \\
  B_{s}^{m\times 1}\\
  \widetilde{B}_{s+1}^{m\times 1}\\
  \vdots\\
  \widetilde{B}_{k+1}^{l\times 1}
  \end{pmatrix},
$$
где матрица $\widetilde{T}$ имеет вид:
$$\widetilde{T} =
  \begin{pmatrix} 
  T & \cdots \\
  0 & E
  \end{pmatrix}.
$$
По индукции получаем верхнюю треугольную матрицу:
$$A=
  \begin{pmatrix} 
    \ E^{m\times m} & \cdots & \cdots & \cdots & \cdots \\
    0  & E^{m\times m} & \cdots &A_{s+1,k}^{m \times m} & A_{s+1,k+1}^{m \times l} \\ 
    \vdots & \vdots & \ddots & \vdots & \vdots \\ 
    \vdots& 0 & \cdots & E^{m\times m} & A_{k,k+1}^{m \times l} \\
    0 & 0 & \cdots & 0 & E^{l \times l} 
  \end{pmatrix},\ 
  B=
   \begin{pmatrix}
    B_{1}^{m\times 1} \\
     \vdots \\
     B_{k}^{m\times 1}\\
     B_{k+1}^{l\times 1}
    \end{pmatrix}.
$$

\newpage
Теперь проведём обратный ход, записывая ответ в вектор $X$:
\begin{align*}
X_{p[k+1]} &= B_{k+1}, \\
X_{p[k]} &= B_k - A_{k,k+1}X_{p[k+1]}, \\
X_{p[k-1]} &= B_{k-1} - A_{k-1,k+1}X_{p[k+1]} - A_{k-1,k}X_{p[k]}, \\
&\dots \\
X_{p[j]} &= B_j - \sum_{i=0}^{k-j} A_{j,k+1-i}X_{p[k+1-i]}.
\end{align*}

\begin{algorithm}
\caption{Основной цикл, $s$-ый шаг}
\begin{algorithmic}[1]
\State $swap(A_{ss},A_{i_0j_0})$
\State // Меняем 1-ю и $i_0$-ю строки в матрице и правой части, а также 1-й и $j_0$-й столбцы матрицы 
\State // (имеются в виду строки и столбцы блоков)
\State $p[s] = j_0$ // Запоминаем перестановку столбцов
\State $C \gets get\_block(A_{ss})$
\State $B \gets inverse(C)$
\State // Если обратной матрицы не существует, программа завершается
\For{$i = s+1$ to $k$}
    \State $G \gets get\_block(A_{si})$
    \State $D \gets mult(B,\ G)$ // Умножение матриц $m\times m \times m\times m$
    \State $set\_block(A_{si},\ D)$
\EndFor
\State $ C \gets get\_block(A_{s,k+1})$
\State $D \gets mult(B,\ C)$ // Умножение матриц $m\times m \times m\times l$
\State $ set\_block(A_{s,k+1},\ D)$
\State // bl = k, если l==0, иначе k+1
\For{$i = s+1$ to $bl$}
    \State $C \gets get\_block(A_{is})$
    \For{$j = s+1$ to $bl$}
        \State $ D \gets get\_block(A_{sj})$
        \State $F \gets get\_block(A_{ij})$
        \State $M \gets mult(C,\ D)$
        \State $G \gets sub(F,\ M)$ // Вычитание матриц
        \State $set\_block(A_{ij},\ G)$
    \EndFor
\EndFor
\State // Если i=k+1 или j=k+1, цикл корректно обрабатывается
\end{algorithmic}
\end{algorithm}

\newpage
\section{Сложность алгоритма}
Алгоритм зависит от двух параметров $n$ и $m$. Сложность выражается как:
$$
S(n,m) = c_1n^3+c_2m^3+c_3n^2m+c_4nm^2+O(n^2+nm+m^2).
$$
Требуется найти константы $c_1,c_2,c_3,c_4$. Заметим, что перестановка строк, столбцов и обратный ход дают вклад $O(n^2+nm+m^2)$. Положим $k = \frac{n}{m}$.

\subsection{Поиск главного элемента}
\begin{align*}
m^3\sum_{s=1}^{k}{(k-s+1)^2} &= m^3\frac{k(k+1)(2k+1)}{6} = \frac{n(n+m)(2n+m)}{6} = \frac{(n^2+nm)(2n+m)}{6} \\
&= \frac{2n^3+n^2m+2n^2m+nm^2}{6} = \frac{1}{3}n^3+\frac{1}{2}n^2m+\frac{1}{6}nm^2.
\end{align*}

\subsection{Обращение главного элемента}
\begin{align*}
m^3\sum_{s=1}^{k}{1} = m^3k = m^2n.
\end{align*}

\subsection{Домножение на строку}
\begin{align*}
2m^3\sum_{s=1}^{k}{(k-s)} = 2m^3\frac{(k-1)k}{2} = m(n-m)n = mn^2-m^2n.
\end{align*}

\subsection{Обнуление столбца}
\begin{align*}
2m^3\sum_{s=1}^{k}{(k-s)^2} &= 2m^3\frac{k(k-1)(2k-1)}{6} = \frac{1}{3}n(n-m)(2n-m) = \frac{1}{3}(n^2-nm)(2n-m) \\
&= \frac{1}{3}(2n^3-n^2m-2n^2m+nm^2) = \frac{2}{3}n^3-n^2m+\frac{1}{3}nm^2.
\end{align*}

Итоговая сложность:
$$
S(n,m) = n^3+\frac{1}{2}n^2m+\frac{1}{2}nm^2+O(n^2+nm+m^2).
$$

Проверка: $S(n,1) = S(n)$, $S(n,n) = S(n)+n^3$.

\end{document}